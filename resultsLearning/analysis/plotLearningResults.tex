\documentclass{article}
\usepackage{pgfplots}
\usepackage{pgfplotstable}
\usepackage{booktabs}
\usepackage{pgffor}
\usepackage{calc}
\usepgfplotslibrary{statistics}
\usepgfplotslibrary{fillbetween}
\usepackage[active,tightpage]{preview}
\PreviewEnvironment{tabular}
\setlength{\PreviewBorder}{4pt}
\pgfplotsset{compat=newest}
\begin{document}

%\newcommand\FSMtype{DFSM}
%\newcommand\Inputs{5}

\message{\FSMtype}
\message{\Inputs}
\def\methods{20}

\newcommand{\getColor}[1]{%
\ifnum 0=#1 orange!90!black\fi
\ifnum 1=#1 orange!80!black\fi
\ifnum 2=#1 orange!70!black\fi
\ifnum 3=#1 orange!60!black\fi
\ifnum 4=#1 orange!50!black\fi
\ifnum 5=#1 yellow!90!black\fi
\ifnum 6=#1 yellow!80!black\fi
\ifnum 7=#1 yellow!70!black\fi
\ifnum 8=#1 pink!90!black\fi
\ifnum 9=#1 pink!70!black\fi
\ifnum 10=#1 cyan\fi
\ifnum 11=#1 olive\fi
\ifnum 12=#1 magenta!95!black\fi
\ifnum 13=#1 magenta!75!black\fi
\ifnum 14=#1 magenta!55!black\fi
\ifnum 15=#1 blue!80!black\fi
\ifnum 16=#1 blue!70!black\fi
\ifnum 17=#1 blue!60!black\fi
\ifnum 18=#1 green!80!black\fi
\ifnum 19=#1 green!70!black\fi
\ifnum 20=#1 green!60!black\fi
}

\pgfplotsset{
	ymajorgrids,
	width=12cm,
	legend entries={
		1: L* - AllPrefixes,
		2: L* - AllSuffixesAfterLastState,
		3: L* - Suffix1by1,
		4: L* - SuffixAfterLastState,
		5: L* - OneSuffix - binary search,
		6: OP - AllGlobally,
		7: OP - OneGlobally,
		8: OP - OneLocally,
		9: DT,
		10: TTT,
		11: Quotient,
		12: GoodSplit - maxDistLen:2 + EQ,
		13: H-learner ES=0 + EQ,
		14: H-learner ES=1 + EQ,
		15: H-learner ES=2 + EQ,		
		16: SPY-learner ES=0 + EQ,
		17: SPY-learner ES=1 + EQ,
		18: SPY-learner ES=2 + EQ,
		19: S-learner ES=0 + EQ,		
		20: S-learner ES=1 + EQ,
		21: S-learner ES=2 + EQ
		},
	legend cell align=left,
	legend to name=learningLegend,
	title={The number of \objective},
	scaled y ticks=false,
	y tick label style={/pgf/number format/.cd,1000 sep={\,},fixed}
}

\begin{tabular}{rl}
\multicolumn{2}{c}{\FSMtype, \Inputs\ inputs: 
3- and 1- quartiles of values of 100 machines per the number of states on the x axis}\\[0.3cm]
\begin{tikzpicture}[baseline]
\def\objective{OQs}
\pgfplotstableread[col sep=tab]{\FSMtype_p\Inputs_\objective.txt}\loadedtable
\begin{axis}[title={The number of Output Queries}]
\foreach \n in {0,...,\methods}{
  \edef\temp{\noexpand\addplot [\getColor{\n},name path=\n_1,smooth,forget plot] table[y=1_\n]\noexpand\loadedtable;
  \noexpand\addplot [\getColor{\n},name path=\n_3,smooth,forget plot] table[y=3_\n]\noexpand\loadedtable;
  \noexpand\addplot [\getColor{\n},fill opacity=0.15] fill between[of=\n_1  and \n_3];}
  \temp
} 
\end{axis}
\end{tikzpicture}
&
\begin{tikzpicture}[baseline]
\def\objective{EQs}
\pgfplotstableread[col sep=tab]{\FSMtype_p\Inputs_\objective.txt}\loadedtable
\begin{axis}[yticklabel pos=right,title={The number of Equivalence Queries}]
\foreach \n in {0,...,\methods}{
  \edef\temp{\noexpand\addplot [\getColor{\n},name path=\n_1,smooth,forget plot] table[y=1_\n]\noexpand\loadedtable;
  \noexpand\addplot [\getColor{\n},name path=\n_3,smooth,forget plot] table[y=3_\n]\noexpand\loadedtable;
  \noexpand\addplot [\getColor{\n},fill opacity=0.15] fill between[of=\n_1  and \n_3];}
  \temp
} 
\end{axis}
\end{tikzpicture}
\\[0.5cm]
\begin{tikzpicture}[baseline]
\def\objective{symbols}
\pgfplotstableread[col sep=tab]{\FSMtype_p\Inputs_\objective.txt}\loadedtable
\begin{axis}[title={The total length of queried sequences (= Symbols)}]
\foreach \n in {0,...,\methods}{
  \edef\temp{\noexpand\addplot [\getColor{\n},name path=\n_1,smooth,forget plot] table[y=1_\n]\noexpand\loadedtable;
  \noexpand\addplot [\getColor{\n},name path=\n_3,smooth,forget plot] table[y=3_\n]\noexpand\loadedtable;
  \noexpand\addplot [\getColor{\n},fill opacity=0.15] fill between[of=\n_1  and \n_3];}
  \temp
} 
\end{axis}
\end{tikzpicture}
&
\begin{tikzpicture}[baseline]
\def\objective{Resets}
\pgfplotstableread[col sep=tab]{\FSMtype_p\Inputs_\objective.txt}\loadedtable
\begin{axis}[yticklabel pos=right,title={The number of Resets}]
\foreach \n in {0,...,\methods}{
  \edef\temp{\noexpand\addplot [\getColor{\n},name path=\n_1,smooth,forget plot] table[y=1_\n]\noexpand\loadedtable;
  \noexpand\addplot [\getColor{\n},name path=\n_3,smooth,forget plot] table[y=3_\n]\noexpand\loadedtable;
  \noexpand\addplot [\getColor{\n},fill opacity=0.15] fill between[of=\n_1  and \n_3];}
  \temp
} 
\end{axis}
\end{tikzpicture}
\\[0.5cm]
\begin{tikzpicture}[baseline]
\def\objective{Exploration}
\pgfplotstableread[col sep=tab]{\FSMtype_p\Inputs_\objective.txt}\loadedtable
\begin{axis}[title={The Size of Observation Tree}]
\foreach \n in {0,...,\methods}{
  \edef\temp{\noexpand\addplot [\getColor{\n},name path=\n_1,smooth,forget plot] table[y=1_\n]\noexpand\loadedtable;
  \noexpand\addplot [\getColor{\n},name path=\n_3,smooth,forget plot] table[y=3_\n]\noexpand\loadedtable;
  \noexpand\addplot [\getColor{\n},fill opacity=0.15] fill between[of=\n_1  and \n_3];}
  \temp
} 
\end{axis}
\end{tikzpicture}
&
\begin{tikzpicture}[baseline]
\def\objective{EE}
\pgfplotstableread[col sep=tab]{\FSMtype_p\Inputs_\objective.txt}\loadedtable
\begin{axis}[yticklabel pos=right,title={Exploration Efficiency (= Size/Symbols)}]
\foreach \n in {0,...,\methods}{
  \edef\temp{\noexpand\addplot [\getColor{\n},name path=\n_1,smooth,forget plot] table[y=1_\n]\noexpand\loadedtable;
  \noexpand\addplot [\getColor{\n},name path=\n_3,smooth,forget plot] table[y=3_\n]\noexpand\loadedtable;
  \noexpand\addplot [\getColor{\n},fill opacity=0.15] fill between[of=\n_1  and \n_3];}
  \temp
} 
\end{axis}
\end{tikzpicture}
\\[0.5cm]
\begin{tikzpicture}[baseline]
\def\objective{seconds}
\pgfplotstableread[col sep=tab]{\FSMtype_p\Inputs_\objective.txt}\loadedtable
\begin{axis}
\foreach \n in {0,...,\methods}{
  \edef\temp{\noexpand\addplot [\getColor{\n},name path=\n_1,smooth,forget plot] table[y=1_\n]\noexpand\loadedtable;
  \noexpand\addplot [\getColor{\n},name path=\n_3,smooth,forget plot] table[y=3_\n]\noexpand\loadedtable;
  \noexpand\addplot [\getColor{\n},fill opacity=0.15] fill between[of=\n_1  and \n_3];}
  \temp
} 
\end{axis}
\end{tikzpicture}
&
\pgfplotslegendfromname{learningLegend}
\end{tabular}

\end{document}

